\documentclass[11pt,a4paper]{article}

% -------------------- Pakete --------------------
\usepackage[ngerman]{babel}
\usepackage[utf8]{inputenc}
\usepackage[T1]{fontenc}
\usepackage{geometry}
\usepackage{tabularx}
\usepackage{enumitem}
\usepackage{titlesec}
\usepackage{setspace}
\usepackage{fontawesome5}
\usepackage[hidelinks]{hyperref}


% -------------------- Layout --------------------
\geometry{
  left=2.5cm,
  right=2.5cm,
  top=2.5cm,
  bottom=2.5cm
}

\setlength{\parindent}{0pt}
\setlength{\parskip}{6pt}
\setlist[itemize]{leftmargin=1.5em}

% -------------------- Abschnittsformat --------------------
\titleformat{\section}
  {\large\bfseries}
  {}
  {0pt}
  {}
  [\titlerule]

% -------------------- Dokument --------------------
\begin{document}

% ==================== KOPF ====================
\begin{center}
    {\LARGE \textbf{Jason Ackermann}}\\[0.6em]

    \faMapMarker*~Würzburg
    \quad
    \faEnvelope~\href{mailto:burgessjason613@gmail.com}{burgessjason613@gmail.com}
    \quad
    \faPhone~+49\,XXX\,XXXXXXX
    \\[0.4em]

    \faGithub~\href{https://github.com/jasonvla}{github.com/jasonvla}
    \quad
    \faLinkedin~\href{https://linkedin.com/in/...}{linkedin.com/in/...}
\end{center}
% ==================== PRAKTISCHE ERFAHRUNGEN ====================
\section{Praktische Erfahrungen}

\begin{tabularx}{\textwidth}{X r}
\textbf{Studentische Hilfskraft}, Würzburg & ab 04/2026 \\
Computergestützte Datenanalyse, Mitarbeit an Forschungsprojekten unter der Leitung von Prof. Dr. Fabian Moss (Institut für Musikforschung, Universität Würzburg) & \\
\end{tabularx}
\begin{itemize}
    \item ...
\end{itemize}

\begin{tabularx}{\textwidth}{X r}
\textbf{Orchestermanagement}, Würzburg & 02/2026 -- 03/2026\\
Praktikum bei Achim Sauer, Einblicke in das Personal- und Datenmanagement des Theaterbetriebs & \\
\end{tabularx}

% ==================== PROJEKTE ====================
\section{Projekte}

\begin{tabularx}{\textwidth}{X r}
\textbf{An Algorithmic Comparison of Bach- and AI-Chorales} & 11/2025 -- heute \\\\
Freiwilliges Forschungsprojekt in Zusammenarbeit mit Prof. Dr. Fabian Moss & \\
\end{tabularx}
\begin{itemize}
    \item Ziel: Umfassendes Verständnis über KI-Kompositionen, Vorstellung der Ergebnisse auf der \textit{ICCCM} (September 2026) 
    \item Auswertung erfolgt durch eigens erstellte Algorithmen in \textit{Python}, unter Zuhilfenahme der \textit{music21} Library
\end{itemize}

% ==================== SCHULE \& STUDIUM ====================
\section{Schule \& Studium}
\begin{tabularx}{\textwidth}{X r}
\textbf{Julius-Maximilians-Universität}, Würzburg & ab 09/2026 \\
\textit{B.A. Digital Humanities} & \\\\
\end{tabularx}
\begin{tabularx}{\textwidth}{X r}
\textbf{Julius-Maximilians-Universität}, Würzburg & 09/2024 -- heute \\
\textit{B.A. Musikwissenschaft (bisher: 1,0)} & \\\\
\end{tabularx}

\begin{tabularx}{\textwidth}{X r}
\textbf{Gymnasium Veitshöchheim}, Veitshöchheim & 09/2016 -- 06/2024 \\
\textit{Abitur (1,2)} & \\\\
\textbf{Eichendorff-Grundschule}, Veitshöchheim & 09/2012 -- 07/2016
\end{tabularx}

% ==================== ENGAGEMENT UND STIPENDIEN ====================
\section{Engagement und Stipendien}

\begin{tabularx}{\textwidth}{X r}
Stipendium der Studienstiftung des deutschen Volkes & 12/2024 -- heute \\
Stipendium von E-Fellows & 06/2024 -- heute \\
Stipendium des MozartLabors (im Rahmen des MozartFests) & 04-/2025 -- 07/2025 \\
Mitglied im Bayerischen Landesjugendorchester & 05/2024 -- 12/2024 \\
Ehrenamtliche Konzerte (Erhalt von Kirchen) & 05/2018 -- 06/2022
\end{tabularx}

% ==================== FÄHIGKEITEN UND INTERESSEN ====================
\section{Fähigkeiten und Interessen}

\textbf{Sprachen:} Deutsch (Muttersprache), Englisch (C1), Französisch (B2), Italienisch (A2)

\textbf{Technische Fähigkeiten:} Python, LaTeX, Git, R (Grundlagen), MS Office

\end{document}
